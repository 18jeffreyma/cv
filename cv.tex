\documentclass{mycv}

\name[Jeffrey Ma]{Cheng XU \small{B.S.}}
\address{California Institute of Technology \\ Pasadena, CA 91125}
\email{jjma@caltech.edu}
\homepage{https://18jeffreyma.github.io}
\github{18jeffreyma}
\linkedin{jma18}

\begin{document}

\maketitle%

\section{Research \\ Interests}

I am interested in Databases, Cryptography, Information Security and Privacy-Aware Computing. My current focuses include:

\begin{itemize}
  \item Authenticated query processing for outsourcing cloud computing.
  \item Searchable blockchain with integrity assurance.
  \item Privacy preserving query processing and access control.
\end{itemize}

\section{Education}

\subsection{California Institute of Technology}[Pasadena, CA]
\vspace{-\parskip}%
\begin{itemize}
  \item B.S. in Computer Science and Business, Economics, and Management \printdate{Nov 2014~--~Feb 2019} \\
  (BEM). Expected Graduation in June 2022.
  \item GPA: 4.2/4.3.
\end{itemize}



\section{Research \\ Experience}

\subsection{Tensor Lab, California Institute of Technology}[Pasadena, CA]
\begin{positions}
  \entry{Undergraduate Researcher}{August 2020~--~Present}
\end{positions}

\begin{itemize}
  \item Advisors: \href{http://tensorlab.cms.caltech.edu/users/anima/}{Prof.~Animashree Anandkumar}, \href{http://shiyuanyuan.site/}{Dr.~Yuanyuan Shi}, \href{https://f-t-s.github.io/}{Florian Sch\"{a}ffer}
  \item Working as an undergraduate researcher at the Tensor Lab on extending and formalizing competitive gradient descent optimization methods to multi-agent, reinforcement learning environments and real world games.
\end{itemize}

\subsection{MRSRL, Stanford University}[Stanford, CA]
\begin{positions}
  \entry{Summer Undergraduate Research Fellowship}{June 2019~--~Dec 2019}
\end{positions}

\begin{itemize}
  \item Advisors: \href{https://profiles.stanford.edu/shreyas-vasanawala}{Prof.~Shreyas Vasanawala}, \href{https://computer-science-and-computer-engineering.uark.edu/directory/index/uid/unakarmi/name/Ukash+Nakarmi/}{Prof.~Ukash Nakarmi}
  \item Selected for a Summer Undergraduate Research Felowship (SURF) at the Magnetic Resonance Systems Research Laboratory (MRSRL), working to develop a machine learning framework to detect motion artifacting in pediatric MRI and provide data informed suggestions to MR technicians, a solution which reduces the inefficient use of high-cost doctor hours on image quality assessment.
  \item First-authored paper accepted to the 2020 IEEE International Symposium on Biomedical Imaging (ISBI). Received full funding through the Hummel-Gray and Housner Funds to attend ISBI.

\end{itemize}

\subsection{The Wall Lab, Stanford University}[Stanford, CA]
\begin{positions}
  \entry{Research Intern}{June 2017~--~July 2018}
\end{positions}

\begin{itemize}
  \item Advisor: \href{https://profiles.stanford.edu/dennis-wall}{Prof.~Dennis P. Wall}, \href{https://ed.stanford.edu/faculty/nhaber}{Prof.~Nick Haber}
  \item Accepted to the Stanford Institute of Medical Research Summer Research Program (SIMR), working as a Research Intern at The Wall Lab to develop a machine-learning classifier for diagnosing Autism Spectrum Disorder based on a patient's ability to recognize emotion and their level of facial engagement.
  \item Research paper accepted to the Journal of Medical Internet Research (JMIR).
\end{itemize}

\section{Industry \\ Experience}

\subsection{Nuro, Machine Learning Infrastructure Team}[Mountain View, CA]
\begin{positions}
  \entry{Incoming Software Engineering Intern}{March 2020}
\end{positions}
\begin{itemize}
  \item Incoming intern at Nuro, an early-stage, goods delivery startup, focused on accelerating the benefits of robotics for everyday life.
\end{itemize}


\subsection{Google Brain, Tensorflow Extended Team}[Mountain View, CA]
\begin{positions}
  \entry{Software Engineering Intern}{June 2020~--~Sept 2020}
\end{positions}

\begin{itemize}
  \item Managers: \href{https://www.linkedin.com/in/\%E5\%98\%89\%E4\%BA\%BF-\%E8\%B5\%B5-8034a8a3/}{Jiayi Zhao},  \href{https://www.linkedin.com/in/ruoyu-liu-b5a67154/}{Ruoyu Liu}
  \item Worked on the TensorFlow Extended (TFX) team under the Google Brain org, on TFX, an end-to-end platform for automatically deploying machine learning (ML) models in production. 
  \item Implemented component and architecture improvements to enable asynchronous component execution and continuous pipeline architecture (ML pipelines which can periodically run and stay updated on continuously arriving batches of data), and explored native support for data streaming sources in TFX.
\end{itemize}

\section{Teaching \\ Experience}

\subsection{CS2: Introduction to Programming Methods}[Caltech]
\begin{positions}
  \entry{Teaching Assistant}{Winter 2021}
\end{positions}

\begin{itemize}
  \item TA for Caltech's CS2 (Introduction to Programming Methods). Topics covered include data structures; implementation and performance analysis of fundamental algorithms; algorithm design principles, in particular recursion and dynamic programming.
\end{itemize}

\subsection{CS24: Computing Systems}[Caltech]
\begin{positions}
  \entry{Head Teaching Assistant}{Fall 2020}
\end{positions}

\begin{itemize}
  \item Head TA for Caltech’s CS24 (Computing Systems), which focuses on a programmer’s view of how computer systems execute programs, store information, and communicate. Topics covered include: machine-level code and its generation by optimizing compilers, performance evaluation and optimization, computer arithmetic, memory organization and management, and supporting concurrent computation.
\end{itemize}

\subsection{CS2: Introduction to Programming Methods}[Caltech]
\begin{positions}
  \entry{Teaching Assistant}{Winter 2020}
\end{positions}

\subsection{CS24: Computing Systems}[Caltech]
\begin{positions}
  \entry{Teaching Assistant}{Fall 2019}
\end{positions}


\section{Skills}

\begin{description}
  \item[Programming] C/C++, Rust, Java, Python, Ruby, Matlab, \LaTeX, Bash, Javascript
  \item[Tools] Vim, Tmux, Git, macOS, Linux
  \item[Languages] English, Mandarin
\end{description}

\section{Publications}%

\textbf{Complete List:}
\href{https://scholar.google.com/citations?user=IemYiGEAAAAJ}{Google Scholar \textsf{\footnotesize [IemYiGEAAAAJ]}}%
{~~$\cdot$~~}%
\href{https://orcid.org/0000-0002-3646-3547}{ORCID \textsf{\footnotesize [0000-0002-3646-3547]}}%

\publications[keyword={selected}]{publications.bib}

% {
% \footnotesize%
% \textsuperscript{\textdagger}These authors contributed equally.
% }


\section{Awards}

\begin{itemize}
%   \item Jack E. Froehlich Memorial Award Nominee (for GPA in the top 5\%) \printdate{2020}
  \item Patrick Hummel and Harry Gray Travel Fund Award \printdate{2020}
  \item George W. Housner Student Discovery Fund Award \printdate{2020}
  \item Gee Family Poster Competition Finalist (for excellence in scientific communication) \printdate{2019}
  \item Andy Grove Scholarship \printdate{2019}
\end{itemize}

\section{Other \\ Activities}



\subsection{Student Faculty Program (SFP) Ambassador}[Caltech]
\begin{itemize}
  \item Head TA for Caltech’s CS24 (Computing Systems), which focuses on a programmer’s view of how computer systems execute programs, store information, and communicate. Topics covered include: machine-level code and its generation by optimizing compilers, performance evaluation and optimization, computer arithmetic, memory organization and management, and supporting concurrent computation.
\end{itemize}

\end{document}
